% !TEX program = pdfLaTeX
% !TEX encoding = UTF-8

% LaTeX resume using res.cls
\documentclass[UTF8,nofonts]{res}
\setlength{\topmargin}{-0.25in}  % Start text higher on the page
\setlength{\textheight}{9.2in}  % increase textheight to fit more on a page
%\setlength{\textwidth}{6.25in}  % increase textheight to fit more on a page
\setlength{\headsep}{0.2in}     % space between header and text
\setlength{\headheight}{12pt}   % make room for header
\usepackage{fancyhdr}  % use fancyhdr package to get 2-line header
\renewcommand{\headrulewidth}{0pt} % suppress line drawn by default by fancyhdr
\lhead{\hspace*{-\sectionwidth}\emph{Wenjie Chen}} % force lhead all the way left
\rhead{\emph{Curriculum Vitae}}  % put page number at right
\cfoot{\thepage}  % the footer is empty
\pagestyle{fancy} % set pagestyle for the document
\usepackage{hyperref}
\hypersetup{colorlinks=true,urlcolor=blue}
\usepackage{etaremune}
\usepackage{ctex}

\begin{document}

\setlength{\parindent}{0pt}

\thispagestyle{empty} % this page does not have a header
\name{WENJIE CHEN (陈文杰)}
\address{
\centerline{\textbf{Email:} wjchen84 AT gmail.com}\\
\centerline{\url{http://wjchen84.github.io/}}}


\begin{resume}
\linespread{1.2}%
\selectfont

\vspace{0.1in}

%\section{FIELD OF INTERESTS}
%\vspace{0.1in}
%Theory and application of dynamic systems, control, and advanced learning; robotic/mechatronic systems. %Applications to industrial robots, wearable assistive robotics, robots for advanced manufacturing.

%Theory and application of control, sensing, and monitoring; mechatronic systems. Applications to industrial robots, human-interactive robots (e.g., exoskeleton), robotics for advanced manufacturing.
%dynamic systems and control, mechatronic systems, wearable assistive robotics, industrial robots, co-robots for advanced manufacturing
%exoskeleton design \& control, real-time implementation, iterative learning control, maximum likelihood estimation, robot dynamic \& kinematic analysis, Linear system, nonlinear control, adaptive control, iterative learning control, sensor fusion, Kalman filter, robust control, hybrid control, motion control, maximum likelihood estimation, human mechatronics, exoskeleton design \& control, brain machine interface, robot dynamic \& kinematic analysis, real-time implementation, robot for manufacturing, neural network learning

%\section{PROJECT EXPERIENCES}
%\vspace{0.1in}
%Industrial robot manipulator, LCD substrate transfer robot, exoskeleton design \& control, brain machine interface, real-time drawing robot, biaxial linear motor stage, virtual reality design, CAD software design, RobotCup small-sized league


\section{教育背景}
\vspace{0.1in}

    \makebox[0.65in][l]{\bf 博士} \makebox[2.5in][l]{\bf 美国加州大学伯克利分校} \makebox[0.75in][l]{\normalfont 机械工程} \hfill 08/2012\\
%    Mechanical Engineering ({\it Specific major: Control; Minor: Optimization, Dynamics})\\
%   GPA: 3.976/4.0, Major: Control (4.0/4.0). Minor: Optimization, Dynamics\\
    {\normalfont 毕业论文:非匹配动力学和非匹配感知下的机器人智能控制\\
    导师:Professor Masayoshi Tomizuka}

    \makebox[0.65in][l]{\bf 硕士} \makebox[2.5in][l]{\bf 美国加州大学伯克利分校} \makebox[0.75in][l]{\normalfont 机械工程} \hfill 05/2009\\
%    Mechanical Engineering ({\it Specific major: Control})\\	
%   GPA: 3.972/4.0, Major: Control (4.0/4.0)\\
    {\normalfont 毕业论文:利用关节传感器信息融合的非直驱传动链的混合自适应摩擦补偿\\
    导师:Professor Masayoshi Tomizuka}

    \makebox[0.65in][l]{\bf 学士} \makebox[2.5in][l]{\bf 浙江大学} \makebox[0.75in][l]{\normalfont 机械电子工程} \hfill 06/2007\\
%    Mechatronic Engineering (\emph{Rank: 1/55})\\
    \makebox[1.25in][l]{\normalfont 排名: 1/55 (专业)} \hfill \normalfont 辅修:工程教育高级班 (从6000多本科生中选拔63人)\\
    {\normalfont 毕业论文:两轴直线驱动平台的协调运动控制\\
    导师:姚斌 教授, 王庆丰 教授}
%    Mechatronic Engineering, ACEE, GPA: 3.87/4.0, 89.35/100
%    Department of Mechanical Engineering & Automation (Mechatronic Engineering)			10/2002 - 06/2007
%    Advanced Honor Class of Engineering Education (63 elites out of more than 6000)			09/2003 - 06/2006
%    GPA:	3.87/4.0,	89.35/100 			RANK:	1/55	(Major)


\section{工作经历}
\vspace{0.1in}
\begin{list}{}{\setlength\leftmargin{0in}\setlength\topsep{0.15in}}

\item \makebox[1.8in][l]{\bf 发那科株式会社} \makebox[2.2in][l]{基础研究所 \hspace{0.1in} 机器人软件研究部} {\bf 主任} \hfill 10/2017 -- {\normalfont 现在}
\begin{itemize}
	\item 下一代机器人软件研发的技术负责人:动作规划和控制,优化和学习等
	\item 技术主导与知名大学的前沿机器人研究合作
\end{itemize}

\item \makebox[1.8in][l]{\bf 发那科株式会社} \makebox[2.2in][l]{机器人研究所 \hspace{0.1in} 学习机器人开发部} {\bf 主任} \hfill 11/2013 -- 10/2017
\begin{itemize}
	\item 下一代机器人概念控制器的研发的技术负责人
	\item 技术主导与知名大学的前沿机器人研究合作
	\item 为现有的学习机器人产品研发提供技术指导和支持
\end{itemize}

\item \makebox[1.8in][l]{\bf 美国加州大学伯克利分校} \makebox[1.65in][l]{机械系统控制实验室} {\bf 博士后研究员}  \hfill 08/2012 -- 10/2013
%    Advisor: Prof. Masayoshi Tomizuka, 
\begin{itemize}
	\item 主导脑机交互研究中的外骨骼机器人的设计和控制
	\item 主导机械臂的智能控制研究,包括控制、运动规划、传感器融合、系统建模和辨识等
\end{itemize}

%\item    {\bf Graduate Student Researcher}, University of California, Berkeley    \hfill Aug. 2008 -- Aug. 2012
%    Advisor: Prof. Masayoshi Tomizuka, Mechanical Systems Control Lab
%    Department of Mechanical Engineering, University of California, Berkeley, CA

%\item    {\bf Graduate Student Instructor}, University of California, Berkeley     \hfill Aug. 2010 -- Dec. 2010
%    EE128/ME134,	Feedback Control Systems
%    Department of Mechanical Engineering & Department of Electrical Engineering & Computer Science,

%\item    {\bf Reader}, University of California, Berkeley     \hfill Aug. 2007 -- May 2008
%    ME132, Dynamic Systems and Feedback \hfill Aug. 2007--May. 2008 \\
%    E7, Introduction to Computer Programming for Scientists and Engineers \hfill Aug. 2007 -- Dec. 2007

%\item    {\bf Senior Consultant Assistant, Internship}      \hfill Jul. 2006 -- Aug. 2006\\
%    Chuanyou Guanghan Honghua Co. Ltd., China

%    {\bf Visiting Student Trainee},      \hfill Jul. 2005 -- Jul. 2005\\
%    Technical Department, Baosteel Group Corp., Shanghai, China

%\item    {\bf Undergraduate Research Assistant}, Zhejiang University, China   \hfill Aug. 2004 -- Jun. 2007
%    Advisor: Prof. Bin Yao, Precision Mechatronic System Lab \hfill Sep. 2006 -- Jun. 2007\\
%    Advisor: Prof. Linyi Gu, Institute of Mechatronic Control Engineering \hfill Aug. 2004 -- Sep. 2006
\end{list}

\section{荣誉和获奖 (节选)}
\vspace{0.2in}

\begin{list}{$\bullet$}{\setlength\leftmargin{0in}\setlength\topsep{0in}}
\item \textbf{最佳应用论文候选}, {第12届国际自动化科学和工程会议 (CASE)} \hfill{2016}
\item \textbf{最佳学生论文候选}, {IEEE/ASME 国际先进智能机电会议(AIM)} \hfill{2015}
\item \textbf{最佳学生论文候选}, {第6届IFAC国际机电系统论坛(MECHATRONICS)} \hfill{2013}
\item \textbf{最佳论文(学习控制主题)}, {ASME国际动力系统与控制会议(DSCC)} \hfill{2012}
\item \textbf{三等奖, Big Ideas @ Berkeley}, \emph{"The PikaPen"} \hfill{04/2012} \\
"社会信息科技" 类别, 125个参赛团队中的5个优胜团队
\item \textbf{Block Grant 奖}, 美国加州大学伯克利分校 \hfill{01/2011}
\item \textbf{蒋震海外留学生奖学金}, 中国 \hfill{2007 -- 2008}\\
每年从中国的海外留学生中选拔10名
\item 浙江省和浙江大学的各种本科生奖项 \hfill{2002 -- 2007}\\
具体请参照: \url{http://wjchen84.github.io/index.html#Honors}
%\item \textbf{Excellent College Graduate}, both Zhejiang University \& Zhejiang Province, China \hfill{Jun. 2007}
%%\item \textbf{Excellent Graduate of Zhejiang University, China} \hfill{Jun. 2007}
%\item \textbf{Excellent Undergraduate Thesis (Design)}, Zhejiang University, China \hfill{Jun. 2007}
%\item \textbf{Weichai Power First Class Fellowship}, China \hfill{Nov. 2006}
%\item \textbf{Ferrotec Fellowship} and \textbf{Fellowship of Tarim Oil Co.}, China \hfill{Nov. 2005}
%%\item \textbf{Fellowship of Tarim Oil Co.}, China \hfill{Nov. 2005}
%\item \textbf{First Prize Scholarship of Excellent Undergraduate} \hfill{2003, 2004, 2006}\\
%Top 3\%, Zhejiang University, China, 3 times
%\item \textbf{Zhuang Caifang Scholarship}, Fujian Province, China \hfill{Jun. 2002}\\
%Top 0.1\%, 200 awardees each year out of over 250,000 high school graduates
\end{list}

%    \begin{tabular}{p{0.7in} p{5in}}
%%      Oct. 2012 & NSF Travel Award, ASME Dynamic Systems and Control Conference (DSCC2012) \\
%%      Jun. 2012 & NSF Travel Award, ASME/ISCIE 2012 International Symposium on Flexible Automation (ISFA2012) \\
%      Apr. 2012 & \textbf{3rd place, Big Ideas @ Berkeley} (Information Technology for Society), \emph{"The PikaPen"} \\
%      Jan. 2011 & Block Grant Award, University of California, Berkeley \\
%%      Oct. 2009 & Student Travel Grant, the 2009 ASME Dynamic Systems and Control Conference (DSCC2009) \\
%      Jun. 2007 & Excellent College Graduate of Zhejiang Province, China \\
%      Jun. 2007 & Excellent Graduate of Zhejiang University, China \\
%      Jun. 2007 & Excellent Undergraduate Thesis (Design) of Zhejiang University, China \\
%      Dec. 2006 & Chiang Chen Overseas Graduate Fellowship (10 awardees each year from China), China \\
%      Nov. 2006 & Weichai Power First Class Fellowship, China \\
%      Nov. 2005 & FERROTEC Fellowship, China \\
%      Nov. 2005 & Fellowship of Tarim Oil Co. China \\
%      2004-2007 & First Prize Scholarship of Excellent Undergraduate, Zhejiang University (Top 3\%, 3 times), China \\
%      Jun. 2002 & Zhuang Caifang Scholarship, Fujian Province, China\\
%    \end{tabular}
%    \begin{itemize}
%    \item Block Grant Award, University of California, Berkeley, Jan. 2011
%    \item Excellent College Graduate of Zhejiang Province, Jun. 2007
%    \item Excellent Graduate of Zhejiang University, Jun. 2007
%    \item Excellent Undergraduate Thesis (Design) of Zhejiang University, Jun. 2007
%    \item Chiang Chen Overseas Graduate Fellowship (10 awardees each year from China), Dec. 2006
%    \item Weichai Power First Class Fellowship, Nov. 2006
%    \item FERROTEC Fellowship, Nov. 2005
%    \item Fellowship of Tarim Oil Co. China, Nov. 2005
%    \item First Prize Scholarship of Excellent Undergraduate, Zhejiang University (Top 3\%, 3 times), 2004-2007
%    \end{itemize}

\section{专业事务}
\vspace{0.1in}
 %   电气电子工程师学会 (IEEE), 会员 \\ % S12, M12
 %   美国机械工程师学会 (ASME), 会员 % S06, M13

    \textbf{专业奖项评委:}\\
    IEEE/IFR IERA Award (机器人和自动化领域的创新和创业精神), 2017
    
    \textbf{研究计划评委:}\\ % 2013
    香港政府研究资助局(RGC)--外部评委 (2013, 2014, 2016, 2017)
 	
    \textbf{杂志编辑委员会:}\\ % 2013
    国际先进机器人系统杂志 (IJARS) 编委会委员 

    \textbf{学术会议委员会:}\\ % 2013
    程序委员会, 2016 ASME 国际柔性自动化论坛 (ISFA);\\
    编委, 2016 美国控制会议 (ACC);\\
    编委, 2015 ASME 国际动力系统和控制会议 (DSCC);\\
    编委, 2015 美国控制会议 (ACC);\\
    主题组织者, 2014 ASME 国际动力系统和控制会议  (DSCC);\\
    程序委员会, 2013 IEEE 国际信息和自动化会议 (ICIA);\\
    程序委员会, 2013 IEEE 国际机器人和仿生学会议 (ROBIO)%;\\
%    Session Chair, 2013 ASME Dynamic Systems and Control Conference (DSCC), "Robotics and Manipulators"

    \textbf{杂志审稿人:}\\
    IEEE Transactions on Robotics (T-RO),
    IEEE Transactions on Industrial Electronics (TIE),
    IEEE/ASME Transactions on Mechatronics (TMECH),
    IEEE Transactions on Control Systems Technology (TCST),
    IEEE Transactions on Automation Science and Engineering (T-ASE),
    ASME Journal of Dynamic Systems, Measurement, and Control (JDSMC),
    Robotics and Computer Integrated Manufacturing (Elsevier-RCIM),
    Robotics and Autonomous Systems (RAS),
    International Journal of Advanced Robotic Systems (IJARS),
    Advanced Robotics (RSJ-AR),
    Asian Journal of Control (AJC),
    Control and Cybernetics,
    Sensors (MDPI Journal),
    Journal of Zhejiang University Science C (Computers \& Electronics) (ZUSC)

%    \begin{itemize}
%    \item IEEE Transactions on Industrial Electronics (TIE)
%    \item IEEE/ASME Transactions on Mechatronics (TMECH)
%    \item Robotics and Computer Integrated Manufacturing (Elsevier-RCIM)
%    \item Asian Journal of Control (AJC)
%    \item Control and Cybernetics
%    \item Sensors (MDPI Journal)
%    \end{itemize}

    \textbf{学术会议审稿人:}\\
    American Control Conference (ACC),
    IEEE Conference on Decision and Control (CDC),
    IEEE/RSJ International Conference on Intelligent Robots and Systems (IROS),
    IEEE International Conference on Robotics and Automation (ICRA),
    IEEE/ASME International Conference on Advanced Intelligent Mechatronics (AIM),
    ASME Dynamic Systems and Control Conference (DSCC),
    ASME International Symposium on Flexible Automation (ISFA),
    IFAC Symposium on Robot Control (SYROCO),
    IEEE International Conference on Information and Automation (ICIA),
    IEEE International Conference on Robotics and Biomimetics (ROBIO)

%    \begin{itemize}
%    \item American Control Conference (ACC)
%    \item IEEE/RSJ International Conference on Intelligent Robots and Systems (IROS)
%    \item IEEE/ASME International Conference on Advanced Intelligent Mechatronics (AIM)
%    \item ASME Dynamic Systems and Control Conference (DSCC)
%    \item ASME International Symposium on Flexible Automation (ISFA)
%    \item IFAC Symposium on Robot Control (SYROCO)
%    \end{itemize}

%10/2006 - Present		Student Member, ASME (American Society of Mechanical Engineers)
%03/2005 - 03/2006		Student Member & Volunteer, IAESTE-CHINA
%02/2005 - 07/2005		Deputy Director, Student Union 2002, College of Mech. & Energy Eng., Zhejiang Univ.
%09/2004 - 10/2005		Director, Supervision Department of Student Work-Study Center, Zhejiang Univ., China
%03/2003 - 07/2005		Monitor, Mechatronic Engineering Class 0202, Zhejiang Univ., China

%\section{RESEARCH EXPERIENCES}
%\vspace{0.1in}
%{\bf NSF-EFRI-M3C: A Hybrid Control Systems Approach to Brain-Machine Interfaces for Exoskeleton Control}
%\hfill  \underline{Postdoc Scholar}, 08/2012 -- 10/2013\\
%University of California, Berkeley \hfill \underline{Graduate Student Researcher}, 01/2012 -- 08/2012\\
%PI: Professor Jose M. Carmena, Professor Masayoshi Tomizuka, Professor Claire J. Tomlin \\
%%    \textbf{Sponsor:} NSF-Emerging Frontiers in Research and Innovation (NSF-EFRI), USA\\
%%    \\Advisor: Prof. Masayoshi Tomizuka. Sponsor: NSF EFRI 10-596
%\underline{Role:} \emph{Technical lead of a sub-group of 1 postdoc, 2 Ph.D. students, and 2 undergrads.} Responsibilities:%Personal major responsibilities:
%\begin{itemize}
%	\item Multi-degree of freedom (e.g., 6-DOF) upper-limb (passive and actuated) exoskeleton design for macaque monkeys in the brain-machine interface (BMI) study.
%	\item Dynamic and kinematic modeling study for multi-DOF exoskeleton system.
%	\item Control scheme (e.g., position control, torque control, impedance control) design for the upper-limb exoskeleton actuation.
%	%    \item Exoskeleton kinematic and dynamic calibration, as well as motion and torque sensing/estimation.
%	\item Collaboration on BMI experiments to address the central research question: Does the brain use motor programs to help it control a highly redundant multi-degree of freedom biomechanical plant such as the arm?
%\end{itemize}
%
%{\bf Intelligent Control of Robot Manipulators for Performance Enhancement}  \\
%Sponsor: FANUC Corporation, Japan \hfill  \underline{Postdoc Scholar}, 08/2012 -- 10/2013\\
%University of California, Berkeley  \hfill \underline{Graduate Student Researcher}, 08/2007 -- 08/2012\\
%%    \textbf{PI:} Professor Masayoshi Tomizuka \hfill \textbf{Sponsor:} FANUC Ltd., Japan\\
%\underline{Role:} \emph{Technical lead of a group of 1 postdoc and 4 Ph.D. students.} Personal major achievements:
%\begin{itemize}
%	\item Dynamic modeling and system identification for multi-joint robots.
%	\item Develop Robot Simulator and Experimentor with customizable software control architecture.
%	\item Sensor fusion for load side (end-effector) state estimation in robots with joint elasticity.
%	\item Friction force identification and compensation in robots with joint elasticity.
%	\item Disturbance rejection controller design, e.g., disturbance observer (DOB) and adaptive robust controller (ARC), in robots with joint elasticity.
%	\item Dual-stage iterative learning control (ILC) in robots with joint elasticity.
%	\item Automatic gain tuning using iterative feedback tuning in indirect drive trains.
%	\item Real-time system implementation with MATLAB \& xPC Target for FANUC M-16\emph{i}B-20 robot, and NI LabVIEW Real-time \& FPGA modules for single-joint robot testbed.
%\end{itemize}
%
%{\bf Development of Design/Analysis and Evaluation Technologies for Vibration Reduction of LCD Substrate Transfer Robot}
%\hfill Sponsor: Hyundai Heavy Industry, South Korea \\
%%    Supported by Hyundai Heavy Industry, South Korea \\
%University of California, Berkeley  \hfill \underline{Graduate Student Researcher}, 01/2011 -- 03/2012\\
%%    \textbf{PI:} Professor Masayoshi Tomizuka \hfill \textbf{Sponsor:} Hyundai Heavy Industry, South Korea\\
%\underline{Role:} \emph{Technical lead of a group of 2 Ph.D. students and 1 visiting industrial fellow.} Personal work:
%\begin{itemize}
%	\item Dynamic modeling and simulation of the LCD substrate transfer robot considering end-effector vibration behavior induced by flexible links.
%	\item Sensor-based learning control and disturbance observer (DOB) design for vibration reduction.
%\end{itemize}
%
%{\bf High-Speed / High-Precision Motion Control of Biaxial Linear Motor}  \\
%Zhejiang University, China  \hfill \underline{Undergraduate Research Assistant}, 09/2006 -- 06/2007
%%    \textbf{PI:} Professor Bin Yao \hfill \textbf{Sponsor:} National Natural Science Foundation of China (NSFC)\\
%%Precision Mechatronic System Lab.
%%    Personal major achievements:
%\begin{itemize}
%	\item Friction force identification and compensation in the biaxial linear motor stage.
%	\item Adaptive robust control (ARC) for biaxial linear motor high-speed / high-precision motion.
%	\item Coordinated motion controller design and comparative study for the biaxial linear motor stage.
%\end{itemize}
%
%{\bf 3D Virtual Reality Software Design for Underwater Manipulator}  \\
%Zhejiang University, China  \hfill \underline{Undergraduate Research Assistant}, 08/2004 -- 09/2006
%%    \textbf{PI:} Professor Linyi Gu \hfill \textbf{Sponsor:} National Natural Science Foundation of China (NSFC)\\
%%    Institute of Mechatronic Control Engineering
%%    Personal major achievements:
%
%\begin{itemize}
%	\item Software design for 3D virtual reality interactive control of the underwater manipulator.
%	\item Independent software system development using Visual C++ \& OpenGL.
%\end{itemize}
%
%%    {\bf Mechanical Design of Football Robot in RoboCup Small-Sized League}  \\
%%    Research Member, Zhejiang University, China  \hfill Sep. 2004 - Oct. 2005
%%%    \\Mechanical Group of RoboCup Small-Sized League, State Key Lab of Industrial Control Technology, Zhejiang University
%%
%%    \begin{itemize}
%%    \item Roller design improvement of the dribbling system in the RoboCup small-sized football robots.
%%    \item Theoretical calculations and experimental validation of the roller design (material, shape and position).
%%    \item Ball-shooting system design for the fifth version of RoboCup small-sized football robots.
%%    \end{itemize}
%
%{\bf Bio-inspired In-Pipe Moving WormBot}  \\
%Zhejiang University, China  \hfill 12/2004 -- 05/2005 \\
%\underline{Role:} \emph{Team leader of 3 undergrads}, the Tenth Mechanical Design Contest, \emph{3rd prize}
%%    \\Supervisor: Prof. Xiangyong Qian, Zhejiang University
%
%\begin{itemize}
%	\item Design \& fabrication of a bio-inspired robot that can move through small bent pipes
%	of certain diameters less than 50cm, based on the principle of worm movement.
%\end{itemize}
%%
%%    {\bf CAD Software - IG-541 Fire Protection Piping System Design }  \\
%%    Team Leader, Student Research Training Program  \hfill Jul. 2004 - Jun. 2005 \\
%%    Zhejiang University, China
%%%    \\Supervisor: Associate Prof. Yingjun Xie, Zhejiang University
%%
%%    \begin{itemize}
%%    \item Design of the algorithm and structure of a cad software - IG-541 fire protection piping system design.
%%    \item Software development using Visual Basic and AutoLisp language.
%%    \item Successful software test in Zhejiang Yongan Fire Protection Company, China.
%%    \end{itemize}


%\section{GRANT PROPOSAL WRITING}
%\vspace{0.2in}
%    \begin{list}{$\bullet$}{\setlength\leftmargin{0in}\setlength\topsep{0in}}
%    \item \textbf{NSF-CMMI: Hybrid Control Strategies for Human Assistive Systems with Variable Dynamic Characteristics} \hfill{2013}\\
%         PI: Prof. Masayoshi Tomizuka \\
%         \emph{Submitted}, $\sim$\$307,076, 3-year \hfill Sponsor: National Science Foundation
%    \item \textbf{NRI-Small: Assistive Exoskeleton for Activities of Daily Living} \hfill{2012}\\
%         PI: Prof. Ruzena Bajcsy; Co-PI: Prof. Masayoshi Tomizuka \\
%         \emph{Submitted}, $\sim$\$1.5-million, 2-laboratory, 5-year \hfill Sponsor: National Robotics Initiative
%    \item \textbf{NSF-CMMI: Control Strategies for Assistive Systems Involving Phase Changes} \hfill{2012}\\
%         PI: Prof. Masayoshi Tomizuka \\
%         \emph{Submitted}, $\sim$\$270,000, 3-year \hfill Sponsor: National Science Foundation
%    \item \textbf{Intelligent Control of Robot Manipulators for Performance Enhancement} \hfill{2008 -- Present}\\
%%         & New proposal every year to FANUC Ltd. Japan\\
%         PI: Prof. Masayoshi Tomizuka \\
%         \emph{Awarded}, yearly proposal, $\sim$\$170,000 per year \hfill Sponsor: FANUC Corporation, Japan
%    \item \textbf{Big Ideas @ Berkeley: The PikaPen} \hfill{2011 -- 2012}\\
%         \emph{Awarded}, \$7,000, competition team member \hfill Sponsor: CITRIS
%    \end{list}


%\section{TEACHING \& MENTORING}
%\vspace{0.1in}
%
%    \textbf{Mentor}, University of California, Berkeley \hfill{2009 -- Present} \\
%    \emph{Supervise research of undergrad and graduate students as well as visiting students (from Spain, Netherland, China, etc.)}
%
%%    \textbf{Graduate Student Instructor (equiv. Teaching Assistant)}, University of California, Berkeley\\
%    \textbf{Teaching Assistant}, University of California, Berkeley\\
%    \emph{Deliver review lectures, discussions, and labs, hold office hours, grade lab reports and exams}
%    \begin{itemize}
%    \item\textbf{Feedback Control Systems} (EE128/ME134, Upper Division) \hfill{Fall 2010}
%    \end{itemize}
%
%%    \textbf{Reader (equiv. Grader)}, University of California, Berkeley\\
%    \textbf{Grader}, University of California, Berkeley\\
%    \emph{Grade problem sets and exams}
%    \begin{itemize}
%    \item\textbf{Dynamic Systems and Feedback} (ME132, Upper Division) \hfill{Fall 2007, Spring 2008}
%    \item\textbf{Introduction to Computer Programming for Scientists \& Engineers} (E7)  \hfill{Fall 2007}
%    \end{itemize}

\section{专利}
\begin{etaremune}
	\item 具备计算传感器的位置和方向的功能的机器人系统 {\it \\JP-2015203902 ({\bf已批准}), US-15/281084 (审批中), CN-201610811511.0 (审批中), DE-102016012065.7 (审批中)}
	\item 物体的姿势计算系统 {\it \\JP-2015177471 ({\bf已批准}), US-15/259118 (审批中), CN-201610814842.X (审批中), DE-102016116404.6 (审批中)}
	\item 具备学习功能的机器人装置 {\it \\JP-2016225207 (审批中), US-15/405190 (审批中), CN-201710025546.6 (审批中), DE-102017000063.8 (审批中)}
	\item 机器人控制装置 {\it \\JP-2016159895 (审批中), US-15/676503 (审批中), CN-201710687596.0 (审批中), DE-102017118276.4 (审批中)}
	\item 具备学习控制功能的机器人系统及其学习控制方法 {\it \\JP-2017026317 (审批中)}
	\item 机器人轨迹自动生成的设备、系统和方法 {\it \\JP-2017077711 (审批中)}
	\item 形状识别和机器人程序生成的设备和方法 {\it \\JP-2017104819 (审批中)}
	\item 机械手控制装置、方法和仿真设备 {\it \\JP-2017129480 (审批中)}
	\item 机器人系统 {\it \\JP-2017164063 (审批中)}
\end{etaremune}

\section{论文}
\vspace{0.1in}
    \textbf{杂志论文} %\small{($*$ denotes corresponding author)}
%    \vspace{-0.1in}
    \begin{etaremune}[start=7]
    \item Junkai Lu, Kevin Haninger, \textbf{Wenjie Chen}, Masayoshi Tomizuka, Suraj Gowda, and Jose M. Carmena, "Design of a Passive Upper Limb Exoskeleton for Macaque Monkeys," \emph{ASME Journal of Dynamic Systems, Measurement, and Control}, 138(11), 111011 (Jul 27, 2016); doi: 10.1115/1.4033837
    \item Pedro Reynoso-Mora, \textbf{Wenjie Chen}, and Masayoshi Tomizuka, "A Convex Relaxation for the Time-optimal Trajectory Planning of Robotic Manipulators along Predetermined Geometric Paths," \emph{Optimal Control Applications and Methods}, vol. 37, no. 6, pp. 1263--1281, Nov./Dec. 2016; doi: 10.1002/oca.2234
    \item \textbf{Wenjie Chen}, Kyoungchul Kong, and Masayoshi Tomizuka, "Dual-Stage Adaptive Friction Compensation for Precise Load Side Position Tracking of Indirect Drive Mechanisms," \emph{Control Systems Technology, IEEE Transactions on}, vol. 23, no. 1, pp. 164--175, Jan. 2015;  doi: 10.1109/TCST.2014.2317776
    \item \textbf{Wenjie Chen}, and Masayoshi Tomizuka, "Dual-Stage Iterative Learning Control for MIMO Mismatched System with Application to Robots with Joint Elasticity," \emph{Control Systems Technology, IEEE Transactions on}, vol. 22, no. 4, pp. 1350--1361, July 2014; doi: 10.1109/TCST.2013.2279652
    \item \textbf{Wenjie Chen}, and Masayoshi Tomizuka, "Direct Joint Space State Estimation in Robots with Multiple Elastic Joints," \emph{Mechatronics, IEEE/ASME Transactions on}, vol. 19, no. 2, pp. 697--706, April 2014; doi: 10.1109/TMECH.2013.2255308
    \item \textbf{Wenjie Chen}, and Masayoshi Tomizuka, "Comparative Study on State Estimation in Elastic Joints," \emph{Asian Journal of Control}, vol. 16, no. 3, pp. 818--829, May 2014; doi: 10.1002/asjc.755%\\ \url{http://dx.doi.org/10.1002/asjc.755}
    \item Jonathan Asensio, \textbf{Wenjie Chen}, and Masayoshi Tomizuka, "Feedforward Input Generation Based on Neural Network Prediction in Multi-Joint Robots," \emph{Journal of Dynamic Systems, Measurement, and Control}, 136(3), 031002, May 2014;   doi:10.1115/1.4025986
%    \item C. Wang$^*$, \textbf{W. Chen}, and M. Tomizuka, "Robot End-effector Sensing with Position Sensitive Detector and Inertial Sensors," \emph{IEEE Transactions on Industrial Electronics}, in preparation
    \end{etaremune}

    \textbf{会议论文} %\small{($*$ denotes corresponding author)}
%    \vspace{-0.1in}
    \begin{etaremune}[start=26]
    \item Yongxiang Fan, Wei Gao, \textbf{Wenjie Chen}, and Masayoshi Tomizuka, "Real-Time Finger Gaits Planning for Dexterous Manipulation," \emph{in Proceedings of the 20th World Congress of the International Federation of Automatic Control (IFAC)}, Toulouse, France, July 9--14, 2017
    \item Chung-Yen Lin, \textbf{Wenjie Chen}, and Masayoshi Tomizuka, "Learning Control for Task Specific Industrial Robots," \emph{in Proceedings of the 55th IEEE Conference on Decision and Control (CDC)}, Las Vegas, USA, December 12--14, 2016
	\item Te Tang, Changliu Liu, \textbf{Wenjie Chen}, and Masayoshi Tomizuka, "Robotic Manipulation of Deformable Objects by Tangent Space Mapping and Non-Rigid Registration," \emph{in Proceedings of the 2016 IEEE/RSJ International Conference on Intelligent Robots and Systems (IROS)}, Deajeon, Korea, October 9--14, 2016
	\item Yu Zhao, \textbf{Wenjie Chen}, Te Tang, and Masayoshi Tomizuka, "Zero Time Delay Input Shaping for Smooth Settling of Industrial Robots," \emph{in Proceedings of the 12th Conference on Automation Science and Engineering (CASE, ISAM 2016)}, Fort Worth, TX, USA, August 21--24, 2016
%	\item Yu Zhao, \textbf{Wenjie Chen}, Xiaowen Yu, Chung-Yen Lin, Te Tang, and Masayoshi Tomizuka, "Accurate Trajectory Tracking of Flexible Joint Robots: A Neuroadaptive Control Approach," \emph{the 2016 IEEE/ASME International Conference on Advanced Intelligent Mechatronics (AIM)}, Banff, Alberta, Canada, July 12--15, 2016, submitted
	\item Te Tang, Hsien-Chung Lin, Yu Zhao, \textbf{Wenjie Chen}, and Masayoshi Tomizuka, "Autonomous Alignment of Peg and Hole by Force/Torque Measurement for Robotic Assembly," \emph{in Proceedings of the 12th Conference on Automation Science and Engineering (CASE, ISAM 2016)}, Fort Worth, TX, USA, August 21--24, 2016 \textbf{(Best Application Paper Finalist)}
	\item Te Tang, Hsien-Chung Lin, Yu Zhao, Yongxiang Fan, \textbf{Wenjie Chen}, and Masayoshi Tomizuka, "Teach Industrial Robots Peg-Hole-Insertion by Human Demonstration," \emph{in Proceedings of the 2016 IEEE/ASME International Conference on Advanced Intelligent Mechatronics (AIM)}, Banff, Alberta, Canada, July 12--15, 2016
    \item Yongxiang Fan, Hsien-Chung Lin, Yu Zhao, Chung-Yen Lin, Te Tang, Masayoshi Tomizuka, and \textbf{Wenjie Chen}, "Object Position and Orientation Tracking for Manipulators Considering Unnegligible Sensor Physics," \emph{in Proceedings of the 2016 International Symposium on Flexible Automation (ISFA)}, Cleveland, USA, August 1--3, 2016
    \item Chung-Yen Lin, Yu Zhao, Masayoshi Tomizuka, and \textbf{Wenjie Chen}, "Path-Constrained Trajectory Planning for Robot Service Life Optimization," \emph{in Proceedings of the 2016 American Control Conference (ACC)}, Boston, MA, USA,July 6--8, 2016
    \item Hsien-Chung Lin, Te Tang, Yongxiang Fan, Yu Zhao, Masayoshi Tomizuka, and \textbf{Wenjie Chen}, "Robot Learning from Human Demonstration with Remote Lead through Teaching," \emph{in Proceedings of the 2016 European Control Conference (ECC)}, Aalborg, Denmark, June 29--July 1, 2016
    \item Hsien-Chung Lin, Te Tang, Masayoshi Tomizuka, and \textbf{Wenjie Chen}, "Remote Lead Through Teaching by Human Demonstration Device," \emph{in Proceedings of the 8th ASME Dynamic Systems and Control Conference (DSCC)}, Columbus, Ohio, USA, October 28--30, 2015
    \item Junkai Lu, Kevin Haninger, \textbf{Wenjie Chen}, and Masayoshi Tomizuka, "Design and Torque-Mode Control of a Cable-Driven Rotary Series Elastic Actuator for Subject-Robot Interaction," \emph{in Proceedings of the IEEE/ASME International Conference on Advanced Intelligent Mechatronics (AIM)}, Busan, Korea, pp. 158--164, July 7--11, 2015 \textbf{(Best Student Paper Finalist)} 
    \item Junkai Lu, \textbf{Wenjie Chen}, Kevin Haninger, and Masayoshi Tomizuka, "A Passive Upper Limb Exoskeleton for Macaques in a BMI Study -- Kinematic Design, Analysis, and Calibration," \emph{in Proceedings of the 7th ASME Dynamic Systems and Control Conference (DSCC)}, San Antonio, Texas, USA, October 22--24, 2014
    \item Kevin Haninger, Junkai Lu, \textbf{Wenjie Chen}, and Masayoshi Tomizuka, "Kinematic Design and Analysis for a Macaque Upper-Limb Exoskeleton with Shoulder Joint Alignment," \emph{in Proceedings of the 2014 IEEE/RSJ International Conference on Intelligent Robots and Systems (IROS)}, Chicago, Illinois, USA, September 14--18, 2014
    \item Yizhou Wang, \textbf{Wenjie Chen}, Masayoshi Tomizuka, and Badr N. Alsuwaidan, "Model Predictive Sliding Mode Control -- for Constraint Satisfaction and Robustness," \emph{in Proceedings of the 6th ASME Dynamic Systems and Control Conference (DSCC)}, Palo Alto, CA, October 21--23, 2013
    \item Chung-Yen Lin, \textbf{Wenjie Chen}, and Masayoshi Tomizuka, "Automatic Sensor Frame Identification in Industrial Robots with Joint Elasticity," \emph{in Proceedings of the 6th ASME Dynamic Systems and Control Conference (DSCC)}, Palo Alto, CA, October 21--23, 2013
    \item Pedro Reynoso-Mora, \textbf{Wenjie Chen}, and Masayoshi Tomizuka, "On the Time-optimal Trajectory Planning and Control of Robotic Manipulators Along Predefined Paths," \emph{in Proceedings of the 2013 American Control Conference (ACC)}, Washington, DC, June 17--19, 2013
    \item Chi-Shen Tsai, \textbf{Wenjie Chen}, Daekyu Yun, and Masayoshi Tomizuka, "Iterative Learning Control for Vibration Reduction in Industrial Robots with Link Flexibility," \emph{in Proceedings of the 2013 American Control Conference (ACC)}, Washington, DC, June 17--19, 2013
    \item Junkai Lu, \textbf{Wenjie Chen}, and Masayoshi Tomizuka, "Kinematic Design and Analysis of a 6-DOF Upper Limb Exoskeleton Model for a Brain-Machine Interface Study," \emph{in Proceedings of the 6th IFAC Symposium on Mechatronic Systems (Mechatronics '13)}, Hangzhou, China, pp. 293--300, April 10--12, 2013 \textbf{(Best Student Paper Finalist)}
    \item Yizhou Wang, \textbf{Wenjie Chen}, and Masayoshi Tomizuka, "Extended Kalman Filtering for Robot Joint Angle Estimation Using MEMS Inertial Sensors,"  \emph{in Proceedings of the 6th IFAC Symposium on Mechatronic Systems (Mechatronics '13)}, Hangzhou, China, pp. 406--413, April 10--12, 2013
    \item \textbf{Wenjie Chen}, and Masayoshi Tomizuka, "Iterative Learning Control with Sensor Fusion for Robots with Mismatched Dynamics and Mismatched Sensing," \emph{in Proceedings of the 2012 ASME Dynamic Systems and Control Conference (DSCC)}, Fort Lauderdale, Florida, USA, pp. 1480--1488, October 17--19, 2012 \textbf{(Best Paper in Session Award)}
    \item Jonathan Asensio, \textbf{Wenjie Chen}, and Masayoshi Tomizuka, "Robot Learning Control Based on Neural Network Prediction," \emph{in Proceedings of the 2012 ASME Dynamic Systems and Control Conference (DSCC)}, Fort Lauderdale, Florida, USA, pp. 1489--1497, October 17--19, 2012
    \item \textbf{Wenjie Chen}, and Masayoshi Tomizuka, "Load Side State Estimation in Robot with Joint Elasticity," \emph{in Proceedings of the 2012 IEEE/ASME International Conference on Advanced Intelligent Mechatronics (AIM)}, Kaohsiung, Taiwan, pp. 598--603, July 11--14, 2012
    \item \textbf{Wenjie Chen}, and Masayoshi Tomizuka, "A Two-Stage Model Based Iterative Learning Control Scheme for a Class of MIMO Mismatched Linear Systems," \emph{in Proceedings of the 2012 ASME International Symposium on Flexible Automation (ISFA)}, St. Louis, Missouri, USA, paper No. ISFA2012--7199, June 18--20, 2012
    \item Cong Wang, \textbf{Wenjie Chen}, and Masayoshi Tomizuka, "Robot End-effector Sensing with Position Sensitive Detector and Inertial Sensors," \emph{in Proceedings of the 2012 IEEE International Conference on Robotics and Automation (ICRA)}, Saint Paul, Minnesota, USA, pp. 5252--5257, May 14--18, 2012
    \item \textbf{Wenjie Chen}, and Masayoshi Tomizuka, "Estimation of Load Side Position in Indirect Drive Robots by Sensor Fusion and Kalman Filtering," \emph{in Proceedings of the 2010 American Control Conference (ACC)}, Baltimore, Maryland, USA, pp. 6852--6857, June 30--July 2, 2010
    \item \textbf{Wenjie Chen}, Kyoungchul Kong, and Masayoshi Tomizuka, "Hybrid Adaptive Friction Compensation of Indirect Drive Trains," \emph{in Proceedings of the 2009 ASME Dynamic Systems and Control Conference (DSCC)}, Hollywood, California, USA, pp. 313--320, October 12--14, 2009
    \end{etaremune}

%    \textbf{Technical Reports}
%    \vspace{-0.1in}
%    \begin{enumerate}
%    \item \textbf{Wenjie Chen}, Pedro Reynoso-Mora, Michael Chan, Cong Wang, Chung-Yen Lin, and Masayoshi Tomizuka, "UCB-FANUC Project Activity Report, June 2011-May 2012," \emph{Technical Report to FANUC Ltd.}, University of California, Berkeley, June 2012
%    \item Daekyu Yun, Chi-Shen Tsai, \textbf{Wenjie Chen}, and Masayoshi Tomizuka, "Development of Design\/Analysis and Evaluation Technologies for Vibration Reduction of LCD Substrate Transfer Robot," \emph{Technical Report to Hyundai Heavy Industries Co., Ltd.}, University of California, Berkeley, February 2012
%    \item \textbf{Wenjie Chen}, Pedro Reynoso-Mora, Michael Chan, Cong Wang, and Masayoshi Tomizuka, "UCB-FANUC Project Activity Report, June 2010-May 2011," \emph{Technical Report to FANUC Ltd.}, University of California, Berkeley, June 2011
%    \item \textbf{Wenjie Chen}, Pedro Reynoso-Mora, Michael Chan, Cheng-Huei Han, and Masayoshi Tomizuka, "UCB-FANUC Project Activity Report, June 2009-May 2010," \emph{Technical Report to FANUC Ltd.}, University of California, Berkeley, June 2010
%    \item Cheng-Huei Han, \textbf{Wenjie Chen}, Pedro Reynoso-Mora, Michael Chan, and Masayoshi Tomizuka, "UCB-FANUC Project Activity Report, June 2008-May 2009," \emph{Technical Report to FANUC Ltd.}, University of California, Berkeley, June 2009
%    \item Cheng-Huei Han, Chun-Chin Wang, \textbf{Wenjie Chen}, and Masayoshi Tomizuka, "UCB-FANUC Project Activity Report, June 2007-May 2008," \emph{Technical Report to FANUC Ltd.}, University of California, Berkeley, June 2008
%    \end{enumerate}

%    \textbf{Theses}
%    \vspace{-0.1in}
%    \begin{enumerate}
%    \item \textbf{Wenjie Chen}, "Intelligent Control of Robots with Mismatched Dynamics and Mismatched Sensing", \emph{Ph.D. Dissertation}, University of California, Berkeley, August 2012
%    \item \textbf{Wenjie Chen}, "Hybrid Adaptive Friction Compensation of Indirect Drive Trains Using Joint Sensor Fusion," \emph{M.S. report}, University of California, Berkeley, May 2009
%    \item \textbf{Wenjie Chen}, "Coordinated Motion Control of Biaxial Linear-Motor-Driven Stage," \emph{B.Eng.~Thesis}, Zhejiang University, China, June 2007 (In Chinese)
%    \end{enumerate}

\section{演讲}
{\it 以上会议论文演讲之外}
%\vspace{0in}
\begin{list}{$\bullet$}{\setlength\leftmargin{0in}\setlength\topsep{0in}}
\item \makebox[0.8in][l]{05/29/2017} "Robotic Learning in Industrial Applications", in \emph{Workshop "Recent Advances in Dynamics for Industrial Applications"}, the 2017 IEEE International Conference on Robotics and Automation (ICRA), Singapore
\item \makebox[0.8in][l]{04/10/2013} "EFRI-M3C: A hybrid control systems approach to brain-machine interfaces for exoskeleton control (Overview)", Qiushi Academy for Advanced Studies, Zhejiang University, China
\item \makebox[0.8in][l]{03/11/2013} "Mechatronic Considerations for Mismatched Robotic Systems", Department of Mechanical Engineering, Carnegie Mellon University
\item \makebox[0.8in][l]{03/04/2013} "Mechatronic Considerations for Mismatched Robotic Systems", Department of Mechanical Engineering, Worcester Polytechnic Institute
\item \makebox[0.8in][l]{02/26/2013} "Mechatronic Considerations for Mismatched Robotic Systems", Department of Mechanical Engineering and Engineering Science, University of North Carolina at Charlotte
\item \makebox[0.8in][l]{08/09/2012} "Intelligent Control of Robots with Mismatched Dynamics and Mismatched Sensing", \emph{Ph.D. seminar}, University of California, Berkeley
\item \makebox[0.8in][l]{03/08/2012} "EFRI-M3C: A hybrid control systems approach to brain-machine interfaces for exoskeleton control (NSF EFRI-M3C 1137267)", \emph{Poster presentation (group work), NSF EFRI Grantees Conference}, Arlington, VA, Mar. 07--09, 2012
\item \makebox[0.8in][l]{02/28/2012} "Estimation in Robots with Mismatched Sensing", \emph{The 1st International Workshop between University of California Berkeley and Keio University}, Berkeley, CA
\item \makebox[0.8in][l]{04/26/2011} "Disturbance Cancellation Schemes for Indirect Drive Robot Manipulator", FANUC Corporation, Japan
\end{list}

%\begin{tabular}{p{0.7in} p{5in}}
%      08/09/2012 & "Intelligent Control of Robots with Mismatched Dynamics and Mismatched Sensing", \emph{Ph.D. seminar}, University of California, Berkeley\\
%%      08/07/2012 & "From Precision Motion Control to Mechatronics Systems Involving Humans", \emph{NI-Week 2012 seminar talk} (group work with X. Chen, C. Wang, A. Oshima, and W. Zhang, content presented by Professor Masayoshi Tomizuka)\\
%%      06/25/2012 & "Issues in Controls of Industrial Robots for Automation", \emph{Invited seminar talk} (group work with C. Wang and C-Y. Lin, content presented by Professor Masayoshi Tomizuka), Xi'an Jiaotong University, China\\
%      03/08/2012 & "EFRI-M3C: A hybrid control systems approach to brain-machine interfaces for exoskeleton control (NSF EFRI 10-596)", \emph{Poster presentation (group work), NSF EFRI Grantees Conference}, Arlington, VA, Mar. 07-09, 2012 \\
%      02/28/2012 & "Estimation in Robots with Mismatched Sensing", \emph{The 1st International Workshop between University of California Berkeley and Keio University}, Berkeley, CA \\
%      04/26/2011 & "Disturbance Cancellation Schemes for Indirect Drive Robot Manipulator", FANUC Ltd., Japan\\
%\end{tabular}


%\section{RELEVANT SKILLS}
%\vspace{0.1in}
%
%    {\bf Programming \& Software}
%    \begin{itemize}
%    \item [] Good knowledge of MATLAB, LabVIEW, C/C++, LaTeX, Microsoft Office %Visual Basic, OpenGL, SQL Language, AutoCAD, HTML
%    \end{itemize}
%
%    {\bf Languages}
%    \begin{itemize}
%    \item [] Native in Chinese; Fluent in English; Elementary in Japanese
%    \end{itemize}

%\section{RELEVANT COURSEWORK}
%\vspace{0.1in}
%
%    {\bf Dynamics \& Control}
%    \begin{itemize}
%    \item [] Dynamic Systems \& Feedback, Advanced Control Systems I \& II, Control of Nonlinear Dynamic Systems, Advanced Robotics, Hybrid System, Real-Time Applications of Mini and Micro Computers, Multivariable Control System Design, Oscillations in Linear Systems, Oscillations in Nonlinear Systems, Intermediate Dynamics
%    \end{itemize}
%
%    {\bf Math \& Optimization}
%    \begin{itemize}
%    \item [] Introduction to Analysis, Mathematical Methods in Engineering, Mathematical Methods for Optimization, Introduction to Convex Optimization, Mathematical Programming II - Nonlinear Programming
%    \end{itemize}

%\section{REFERENCES}
%\vspace{0.1in}
%    \begin{tabular}{p{3.25in} p{3.25in}}
%    {\bf Professor Masayoshi Tomizuka} & {\bf Professor Ruzena Bajcsy}\\
%    \emph{Cheryl \& John Neerhout, Jr., Distinguished Professor} & \emph{Director Emerita of CITRIS}\\
%    Department of Mechanical Engineering & Department of EECS\\
%    University of California, Berkeley & University of California, Berkeley\\
%    5100B Etcheverry Hall, Mailstop 1740 & 719 Sutardja Dai Hall\\
%    Berkeley, CA 94720-1740 & Berkeley, CA 94720\\
%    Phone: 510-642-0870 & Phone: 510-642-9423\\
%    Email: tomizuka@me.berkeley.edu & Email: bajcsy@eecs.berkeley.edu
%    \end{tabular}
%
%    \begin{tabular}{p{3.25in} p{3.25in}}
%    {\bf Professor J. Karl Hedrick} & {\bf Professor Bin Yao}\\
%    \emph{James Marshall Wells Academic Chair Professor} & {School of Mechanical Engineering}\\
%    Department of Mechanical Engineering & Purdue University\\
%    University of California, Berkeley & 585 Purdue Mall\\
%    5104 Etcheverry Hall, Mailstop 1740 & West Lafayette, IN 47907-2040\\
%    Berkeley, CA 94720-1740 & Phone: 765-494-7746(O)\\
%    Phone: 510-642-2482 & 765-538-3389 (On-line)\\
%    Email: khedrick@me.berkeley.edu & Email: byao@purdue.edu
%    \end{tabular}
%
%    \begin{tabular}{p{3.25in} p{3.25in}}
%    {\bf Professor Ronald S. Fearing} (Teaching) & \\
%    Department of EECS & \\
%    University of California, Berkeley & \\
%    725 Sutardja Dai Hall & \\
%    Berkeley, CA 94720-1770 & \\
%    Phone: 510-642-9193 & \\
%    Email: ronf@eecs.berkeley.edu &
%    \end{tabular}

%    {\bf Professor Masayoshi Tomizuka}\\
%    \emph{Cheryl and John Neerhout, Jr., Distinguished Professor}\\
%    Department of Mechanical Engineering\\
%    University of California, Berkeley\\
%    5100B Etcheverry Hall, Mailstop 1740\\
%    Berkeley, CA 94720-1740\\
%    Phone: 510-642-0870\\
%    Email: tomizuka@me.berkeley.edu
%
%    {\bf Professor Ruzena Bajcsy}\\
%    \emph{Director Emerita of CITRIS (the Center for Information Technology Research in the Interest of Science)}\\
%    Department of Electrical Engineering and Computer Sciences\\
%    University of California, Berkeley\\
%    719 Sutardja Dai Hall\\
%    Berkeley, CA 94720\\
%    Phone: (510) 642-9423\\
%    Email: bajcsy@eecs.berkeley.edu

\vspace{0.3in}
\centerline{\footnotesize Updated on November 1, 2017}
\centerline{\footnotesize \url{http://wjchen84.github.io/}}

\end{resume}
\end{document}
